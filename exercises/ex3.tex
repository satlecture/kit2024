\documentclass{article}

\usepackage{listings}
\usepackage{graphicx}
\usepackage{tabularx}
\usepackage{tikzsymbols}
\usepackage{hyperref}
\usepackage{amsmath}

\usepackage{titlesec}
\titleformat*{\section}{\large\bfseries}

\definecolor{myblue}{HTML}{0D3B66}
\definecolor{myred}{HTML}{6E0E0A}
\definecolor{mypink}{HTML}{F7B2B7}

\newcommand{\vars}[1]{\textsf{vars} (#1)}
\newcommand{\lits}[1]{\textsf{lits} (#1)}
\newcommand{\clss}[1]{\textsf{clss} (#1)}

\newcommand{\highl}[1]{\textcolor{myblue}{#1}}
\newcommand{\highlo}[1]{\textcolor{myred}{#1}}

\begin{document}
\vspace*{-1em}
\exhead{3}{2024-06-04}{2024-06-18}

\section{CDCL ($7$ Points)}
Simulate the CDCL algorithm by hand on the formula $F$.
Draw the implication graph for each conflict and learn the 1-UIP clause.
Select branching literals in the order $x_1, x_2, x_3, \dots$
\vspace*{-1ex}
\begin{align*}
F = \bigl\{ &\{ x_1, x_{13} \}, \{ \overline x_1, \overline x_2, x_{14} \}, \{ x_3, x_{15} \}, \{ x_4, x_{16} \}, \\
            &\{ \overline x_5, \overline x_3, x_6 \}, \{ \overline x_5, \overline x_7 \}, \{ \overline x_6, x_7, x_8 \}, \{ \overline x_4, \overline x_8, \overline x_9 \}, \\
            &\{ \overline x_1, x_9, \overline x_{10} \}, \{ x_9, x_{11}, \overline x_{14} \}, \{ x_{10}, \overline x_{11}, x_{12} \}, \{ \overline x_2, \overline x_{11}, \overline x_{12} \} \bigr\}
\end{align*}

\section{Variable Elimination ($3$ Points)}
For a gate encoding $E$ with output $x$ in a formula $F = E \cup R$,
we simplified the resolvents $(E_x \cup R_x) \otimes (E_{\overline x} \cup R_{\overline x})$ by
$S := (E_x \otimes R_{\overline x}) \cup (R_x \otimes E_{\overline x})$, dropping both
$R_x \otimes R_{\overline x}$ and $E_x \otimes E_{\overline x}$.
Show that the clauses in $R_x \otimes R_{\overline x}$ can be derived from $S$ by resolution.
You can assume that $E$ encodes a binary AND gate.

\section{Variable Elimination ($2 \times 2$ Points)}
Let the formula $S$ with gate encodings $E_1$ and $E_2$ be given.
Apply variable elimination for gates for variables $a$ and $r$.
Give the clause sets after each elimination step.
Try the following two strategies.
\vspace*{-1ex}
\begin{enumerate}\setlength{\itemsep}{0pt}
\item Eliminate variable $a$ first, and then $r$ if possible.
\item Eliminate variable $r$ first, and then $a$ if possible.
\end{enumerate}
\vspace*{-1ex}
\begin{align*}
S = \bigl\{ \underbrace{\{ \lnot x, \lnot y, a \}, \{ x, \lnot a \}, \{ y, \lnot a \}}_{E_1},
\underbrace{\{ \lnot a, r \}, \{ \lnot z, r \}, \{ a, z, \lnot r \}}_{E_2}, \{ a, z, r \},
\{ \lnot a, \lnot r \} \bigr\}
\end{align*}

\section{Blocked Clauses ($3\times 3$ Points)}
If Blocked Clause Elimination (BCE) reduces a formula $F$ to the empty formula then $F$ is called a blocked set.
Prove the following statements.
\vspace*{-1ex}
\begin{enumerate}\setlength{\itemsep}{0pt}
	\item Any formula $F$ can be partitioned into two blocked sets $S$ and $L$ such that $F = S\cup L$.
    Design a linear algorithm that produces $L$ and $S$ from $F$.
	\item Blocked sets are not closed unter resolution. 
    If $F$ is a blocked set then $F \cup C_1 \otimes C_2$, where $C_1,C_2 \in F$ may not be a blocked set anymore.
	\item Blocked sets are not closed unter partially assigning variables.
    If $F$ is a blocked set then $F_{x=v}$ (the result of assigning $v$ to $x$ and subsequent simplification) may not be a blocked set anymore.
\end{enumerate}

\section{Hidoku Competition ($12$ Points)}

\newcounter{row}
\newcounter{col}
\newcommand\setrow[4]{
  \setcounter{col}{1}
  \foreach \n in {#1, #2, #3, #4} {
    \edef\x{\value{col} - 0.5}
    \edef\y{4.5 - \value{row}}
    \node[anchor=center] at (\x, \y) {\n};
    \stepcounter{col}
  }
  \stepcounter{row}
}

Hidoku a.k.a Hidato a.k.a Number Snake is a logic puzzle where the goal is to fill a grid with consecutive numbers that connect horizontally, vertically, or diagonally. The grid is rectangular and some of the cells are pre-filled.\\[1ex]

\begin{tikzpicture}[scale=.5]
\begin{scope}
    \draw (0, 0) grid (4, 4);
    \setcounter{row}{1}
    \setrow {1}{}{}{5}
    \setrow {}{7}{}{}
    \setrow {}{}{}{14}
    \setrow {}{}{16}{}
    \node[anchor=center] at (2, -0.9) {Unsolved Hidoku};
\end{scope}
\begin{scope}[xshift=8cm]
    \draw (0, 0) grid (4, 4);
    \setcounter{row}{1}
    \setrow {1}{3}{4}{5}
    \setrow {2}{7}{6}{13}
    \setrow {8}{11}{12}{14}
    \setrow {9}{10}{16}{15}
    \node[anchor=center] at (2, -0.9) {Solved Hidoku};
\end{scope}
\begin{scope}[xshift=16cm]
    \draw (0, 0) grid (4, 4);
    \setcounter{row}{1}
    \setrow {1}{}{}{5}
    \setrow {}{}{}{}
    \setrow {2}{}{}{14}
    \setrow {}{}{16}{}
    \node[anchor=center] at (2, -0.9) {Unsolvable Hidoku};
\end{scope}
\end{tikzpicture}
~\\[1ex]
The input is a string, which represents a Hidoku puzzle.
The first two numbers are the width and heigth of the grid followed by the values separated by commas, rows are separated by semicolons, 0 represents an empty cell.
For the example above it looks as follows.
\vspace*{-1ex}
\begin{align*}
4,4:1,0,0,5;0,7,0,0;0,0,0,14;0,0,16,0
\end{align*}%
%
The output is a string, which represents the solution of the given Hidoku puzzle.
A Hidoku puzzle may be unsatisfiable, in that case the output is \texttt{sol:UNSAT}, otherwise the solution is given in the same format as the input.
For the example above it looks as follows.
\vspace*{-1ex}
\begin{align*}
sol:1,3,4,5;2,7,6,13;8,11,12,14;9,10,16,15;
\end{align*}%
%
Implement a SAT based Hidoku solver.
For a working solver you get 12 points.
The fastest solver will receive a \highl{bonus of 12 points}.
Some examples for testing can be found online at 
\url{https://satlecture.github.io/kit2024/exercises/hidoku/hidokus.txt}

\end{document}