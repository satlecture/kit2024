\documentclass[t]{sdqbeamer}
%\documentclass[c]{sdqbeamer} 

\usepackage{listings}
\usepackage{graphicx}
\usepackage{tabularx}
\usepackage{multirow}
\usepackage{multicol}
\usepackage{tabulary}
\usepackage{colortbl}
\usepackage{tikzsymbols}
\usepackage{tikz}
\usetikzlibrary{positioning,fit,shapes}
\usepackage[lined,linesnumbered,ruled,noend]{algorithm2e}
\usepackage{bm}
\usepackage{enumitem}
\setlist[enumerate]{label*=\bf\alph*),ref=\alph*}
\setlist[itemize]{label=\textbullet}

\hypersetup{
	colorlinks=true,
	urlcolor=kit-orange
}

% set sdqbeamer options
\titleimage{blender-render}
\groupname{Algorithm Engineering}
\grouplogo{ae}
\selectlanguage{english}

% define title etc.pp.
\title[SAT Solving]{Practical SAT Solving}
\subtitle{Lecture 7}
\author{\underline{Markus Iser}, Dominik Schreiber, Tom\'a\v{s} Balyo}
\date{June 3, 2024}

% Existing KIT colors: kit-green, kit-blue, kit-red, kit-gray, kit-orange, kit-lightgreen, kit-brown, kit-purple, kit-cyan
% configure appearance
\setbeamercolor{block title}{bg=kit-blue}
\setbeamercolor{block body}{bg=kit-blue!10}
\setbeamercolor{block title example}{bg=kit-orange}
\setbeamercolor{block body example}{bg=kit-orange!10}
\setbeamertemplate{itemize item}{\color{kit-gray}\textbullet}
\setbeamertemplate{itemize subitem}{\color{kit-gray}\textbullet}
\setbeamercolor{item projected}{bg=kit-gray, fg=kit-gray}
\renewcommand{\insertnavigation}[1]{} % remove navigation bar

% define commands
\definecolor{myblue}{HTML}{0D3B66}
\definecolor{myred}{HTML}{6E0E0A}
\definecolor{mypink}{HTML}{F7B2B7}

\newcommand{\vars}[1]{\textsf{vars} (#1)}
\newcommand{\lits}[1]{\textsf{lits} (#1)}
\newcommand{\clss}[1]{\textsf{clss} (#1)}

\newcommand{\highl}[1]{\textcolor{myblue}{#1}}
\newcommand{\highlo}[1]{\textcolor{myred}{#1}}


\begin{document}
\begin{frame}
	\thispagestyle{empty}
	\titlepage
\end{frame}


\begin{frame}{Recap}
    \begin{block}{Lecture 6: Modern SAT Solving 2}
		\begin{itemize}\setlength{\itemsep}{1ex}
			\item Efficient Unit Propagation
			\item Clause Forgetting
			\item Modern Decision Heuristics: VSIDS \& Co.
		\end{itemize}
    \end{block}
    % \pause
	\begin{block}{Today}
        Preprocessing
	\end{block}
\end{frame}


\begin{frame}{Conflict-driven Clause Learning (CDCL) Algorithm}
\vspace*{-1em}
\begin{columns}[T]
\begin{column}{.3\linewidth}
~\\[1em]
\highl{Last Time}
\begin{itemize}%\setlength{\itemsep}{1em}
    \item Efficient Unit Propagation
    \item Clause Forgetting
    \item Modern Decision Heuristics
\end{itemize}
\highlo{Today}
\begin{itemize}
    \item Preprocessing
\end{itemize}
\end{column}
\begin{column}{.6\linewidth}
\begin{algorithm}[H]
    \DontPrintSemicolon
    \caption{CDCL(CNF Formula $F$, \&Assignment A $\leftarrow \emptyset$)}
    
    \SetKwFunction{propagation}{\highl{UNIT PROPAGATION}}
    \SetKwFunction{branching}{\highl{BRANCHING}}
    \SetKwFunction{conflictanalysis}{CONFLICT-ANALYSIS}
    \SetKwFunction{restart}{RESTART}
    \SetKwFunction{cleanup}{\highl{CLEANUP}}
    \SetKwFunction{preprocessing}{\highlo{PREPROCESSING}}

    \SetKwData{UNSAT}{UNSAT}
    \SetKwData{SAT}{SAT}
    
    \lIf {not \preprocessing} {
        \Return \UNSAT
    }
    \While {A is not complete} {
        \propagation\;
        \If {A falsifies a clause in $F$} {
            \lIf {decision level is $0$} {
                \Return \UNSAT
            }
            \Else {
                (clause, level) $\leftarrow$ \conflictanalysis\;
                add clause to $F$ and backtrack to level\;
                \textbf{continue}\;
            }
        }
        \lIf {\restart} {
            backtrack to level $0$
        }
        \lIf {\cleanup} {
            forget some learned clauses
        }
        \branching\;
    }
    \Return \SAT\;
\end{algorithm}
\end{column}
\end{columns}
\end{frame}


\begin{frame}{Preprocessing}
Preprocessing takes place between problem encoding and its solution.\\[1ex]
\begin{block}{Preprocessing is \dots}
\begin{itemize}
    \item a form of \highlo{reencoding} a problem: to fix bad encodings
    \item a form of \highlo{reasoning} itself: inprocessing
\end{itemize}
\end{block}
\textbf{Conjecture:} Smaller problems are easier to solve 
$\Longrightarrow$ Try to reduce the size of the formula.\\[1ex]
\begin{exampleblock}{Classic Preprocessing Techniques}
\begin{itemize}\setlength{\itemsep}{1ex}
    \item Subsumption
    \item Self-subsuming Resolution
    \item (Bounded) Variable Elimination (BVE)
\end{itemize}
\end{exampleblock}
\end{frame}


\begin{frame}{Preprocessing: Subsumption}
A clause $C$ is subsumed by $D$ iff $D \subseteq C$.\\[1ex]
Subsumed clauses can be removed from the formula without changing satisfiability: $\forall D \subseteq C, D \models C$
\begin{exampleblock}{Example}
    $\{a, b\}$ subsumes $\{a, b, c\}$ and $\{a, b, d\}$
\end{exampleblock}
\pause
\begin{block}{Implementation 1: Forward Subsumption}
Select clause $C$ and \highlo{check if it is subsumed} by any other clause $D \subseteq C$.
\begin{itemize}\setlength{\itemsep}{1ex}
    \item Temporarily mark all literals in $C$ as unsatifisfied, and use \highl{one-watched literal data-structure} to find subsumed clauses
    \item \textbf{Optimization 1:} Watch literals with the fewest occurrences
    \item \textbf{Optimization 2:} Keep literals sorted and perform merge-sort style subset check
\end{itemize}
\end{block}
\end{frame}


\begin{frame}{Preprocessing: Subsumption}
\begin{block}{Implementation 2: Backward Subsumption}
\begin{columns}[T]
\begin{column}{.45\linewidth}~\\
    Select clause $D$ and \highlo{check if it subsumes} any other clause $C \supseteq D$.\\[1ex]
    \highl{Learned clauses} are never subsumed but \highl{can subsume other clauses}, e.g., recently learned clauses
    \begin{itemize}\setlength{\itemsep}{1ex}
        \item \textbf{Optimization 1:} Check the clauses of the variable with the fewest occurrences (scales to large formulas)
        \item \textbf{Optimization 2:} Use signatures to skip the majority of subsumption checks (cf. Bloom filters)
    \end{itemize}
\end{column}
\begin{column}{.5\linewidth}
    \begin{algorithm}[H]
        \DontPrintSemicolon
        \caption{Signature-based Subsumption Check}
        \tcp{Initialization:}
        \For {$clause \in formula$} {
            clause.signature = 0\;
            \For {$lit \in *clause$} {
                clause.signature |= $1\text{ull} << \bigl(\mathop{id}(lit) \% 64\bigr)$\;
            }
        }
        \BlankLine
        \tcp{Subsumption Check:}
        \If {$D.signature~ \& \mathop{invert}(C.signature) == 0$} {
            \tcp{Check if $D$ subsumes $C$}
        }
    \end{algorithm}
\end{column}
\end{columns}
\end{block}
\end{frame}
    
    
\begin{frame}{Preprocessing: Self-Subsuming Resolution}
Applicable if the resolvent of $C$ and another clause $D$ subsumes $C$.\\[1ex]
If $C \otimes_x D \subseteq C$ then $C$ can be replaced by $C \otimes_x D$.
\begin{exampleblock}{Example}
    Let $\otimes_f$ be the resolution operator on variable $f$.\\[1ex]
    $C := \{ \lnot b, \lnot e, {\color{myblue} f}, \lnot h \}$ \qquad
    $D := \{ \lnot b, \lnot e, {\color{myblue} \lnot f} \}$ \qquad
    $E := C \otimes_f D = \{ \lnot b, \lnot e, \lnot h \}$ \\[1ex]
    $\bm\longrightarrow$ Replace $C$ by $E$ (``clause strengthening'')
\end{exampleblock}
\begin{block}{Implementation}
\begin{itemize}\setlength{\itemsep}{1ex}
    \item \textbf{Integrate with subsumption:} Allow at most one literal of D to occur negated in C
    \item \textbf{Variant:} On-the-fly subsumption/strengthening of reason clauses during conflict analysis
\end{itemize}
\end{block}
\end{frame}
    
    
\begin{frame}{Preprocessing: Bounded Variable Elimination}
Let $S_x, S_{\overline x} \subseteq F$ be the sets of all clauses containing $x$ resp. ${\overline x}$, and let $R = \{ C \otimes_x D ~|~ C \in S_x, D \in S_{\overline x} \}$ be the set of all resolvents on $x$.
%
The formulas $F$ and \highl{$F' := (F \setminus (S_x \cup S_{\overline x})) \cup R$} are \highlo{equisatisfiable} but not equivalent.\\[1ex]
%
Most important preprocessing technique in practical SAT solving
\begin{block}{Bounded Variable Elimination (BVE)}
Eliminate variable only if the formula \highlo{size does not increase} (too much).
\begin{itemize}\setlength{\itemsep}{1ex}
    \item \textbf{Note 1:} Variables of removed clauses have to be rescheduled for further elimination attempts
    \item \textbf{Note 2:} Resolvent can trigger further subsumptions and vice versa
    \item Particularly effective in presence of functional definitions (cf. Tseitin encoding)
    \item \textbf{Variant:} Incrementally Relaxed BVE: Increase bound each round if formula size did not increase too much
    \item \textbf{Optimizations:} Perform check only for bounded clause size, resolvent size, or variable occurrence count
\end{itemize}
\end{block}
\end{frame}

    
\begin{frame}{Preprocessing: Blocked Clause Elimination (BCE)}
A clause $\{ x \} \cup C$ is blocked in $F$ by $x$ if either $x$ is \highlo{pure} in $F$ or\\
for every clause $\{ \lnot x \} \cup D$ in $F$ the resolvent $C \cup D$ is a \highlo{tautology}.\\[1ex]
$\rightarrow$ Dead ends in the resolution graph, no proof beyond this point.\\[1ex]
Blocked clause elimination (BCE) has a unique fixpoint, and \highlo{preserves satisfiability}.
\begin{exampleblock}{Example}
$F := (a \lor b) \land (a \lor \lnot b \lor \lnot c) \land (\lnot a \lor c)$\\[1ex]
First clause is not blocked, second is blocked by both $a$ and $\lnot c$, third is blocked by $c$.
\end{exampleblock}
\begin{itemize}\setlength{\itemsep}{1ex}
    \item Effectiveness of BVE can be increased by interleaving it with BCE
    \item Relationship with \href{https://doi.org/10.1007/s10817-011-9239-9}{circuit-level simplification techniques}
    \item \textbf{Generalization: Covered Clauses}\\
    A clause is covered if it can be \highlo{turned into a blocked clause} by adding a covered literal.
    A literal $x$ is covered by a clause $C$, if it contains a literal $y$ such that all non-tautological resolvents of $C$ on $y$ contain $x$.
\end{itemize}
\end{frame}


\begin{frame}{Preprocessing: Solution Reconstruction}
Many preprocessing techniques remove clauses or variables from a formula in a mere satisfiability-preserving way, such that the solution to the preprocessed formula might not be a solution to the original formula.\\[1ex]
\begin{block}{Reconstruction Algorithm}
\begin{columns}[T]
\begin{column}{.35\linewidth}
~\\
Keep track of eliminated variables (BVE) and clauses (BCE) in a solution reconstruction stack $S$, and if a model is found, use it to reconstruct a solution to the original formula.
\end{column}
\begin{column}{.55\linewidth}
\begin{algorithm}[H]
\caption{Solution Reconstruction}
\KwData{Assignment $A$, Stack $S$}
\While {$S$ is not empty} {
    remove the last literal-clause pair $(l,C)$ from $S$\;
    \If {$C$ is not satisfied by $A$} {
        $A := (A \setminus \{l = 0\}) \cup \{l = 1\}$
    }
}
\BlankLine
If variables remain unassigned in $A$, then assign them an arbitrary value.
\end{algorithm}
\end{column}
\end{columns}
\end{block}
\end{frame}


\begin{frame}{Recap.}
\begin{block}{Preprocessing: Classic Techniques}
\begin{itemize}\setlength{\itemsep}{1ex}
    \item Subsumption and Self-subsuming Resolution
    \item Bounded Variable Elimination
    \item Blocked Clause Elimination
    \item Solution Reconstruction
\end{itemize}
\end{block}
\begin{block}{Next Up}
Relationship between preprocessing techniques and gate encodings
\end{block}
\end{frame}


\begin{frame}{Preprocessing: Relationship with Gate Encodings}
Tseitin encoding $E$ of a gate with output $o$, function $g$, and input literals $x_1, \dots, x_n$: $$E \equiv o \leftrightarrow g(x_1, \dots, x_n)$$ 
\vspace*{-1em}
\begin{block}{Properties of Gate Encodings}
Let a Tseitin encoding $E \equiv o \leftrightarrow g(x_1, \dots, x_n)$ be given, and let $A(X) := \{ T \cup \{ \overline x \mid x \in X \setminus T \} \mid T \in 2^X \}$ denote the set of all assignments to variables in $X$.\\[1ex]
For each input assignment $I \in A(\{x_1, \dots, x_n\})$, \\[1ex]
\begin{enumerate}\setlength{\itemsep}{1ex}
\item there exists \highlo{at least one} output assignment $O \in \{o, \overline o\}$ such that $I \cup O \models E$ \quad (left-totality)
\item there exists \highlo{at most one} output assignment $O \in \{o, \overline o\}$ such that $I \cup O \models E$ \quad (right-uniqueness)\\[1ex]
\end{enumerate}
$\bm\rightarrow$ The output is uniquely determined by the input, such that either $I, o \models E$ and $I, \overline o \not\models E$ or vice versa.
\end{block}
\end{frame}


\begin{frame}{Preprocessing: Relationship with Gate Encodings}
From the left-totality it follows that a Tseitin encoding $E$ is a satisfiable set of blocked clauses.\\[1em]
\begin{block}{Left-Totality of Gate Encodings}
Let a \highlo{Tseitin encoding $E \equiv o \leftrightarrow g(x_1, \dots, x_n)$} be given, it holds that\\[1em]
\begin{enumerate}\setlength{\itemsep}{1ex}
\item for \highlo{each clause $C \in E$}, either $o \in C$ or $\overline o \in C$\\[1ex]
\textbf{Proof:} The existence of a clause $C \in E$ such that $o \not\in \vars{C}$ would contradict left-totality, because the assignment falsifiying $C$, falsifies $E$ for any assignment to $o$.
\item and \highlo{all resolvents $R \in E_o \otimes_o E_{\overline o}$} are tautological.\\[1ex]
\textbf{Proof:} The existence of a non-tautological resolvent $R \in E_o \otimes_o E_{\overline o}$ would contradict left-totality, because $E \models R$ and $o \not\in \vars{R}$, such that the assignment falsifying $R$, falsifies $E$ for any assignment to $o$.
\end{enumerate}
\end{block}
\end{frame}


\begin{frame}{Preprocessing: Relationship with Gate Encodings}
From the left-totality it follows that a Tseitin encoding $E$ is a satisfiable set of blocked clauses.\\[1em]
\begin{example}[Tseitin encoding $E \equiv o \leftrightarrow x \land y$]
Let a Tseitin encoding $E := \bigl\{\{ \lnot o, x \}, \{ \lnot o, y \}, \{ o, \lnot x, \lnot y \} \bigr\} \equiv o \leftrightarrow x \land y$ be given, it holds that\\[1em]
\begin{enumerate}\setlength{\itemsep}{1ex}
\item all resolvents in $E_o \otimes_o E_{\overline o} = \bigl\{ \{ x, \lnot x, \lnot y \}, \{ y, \lnot x, \lnot y \} \bigr\} \equiv \top$ are tautological,
\item and Blocked Clause Elimination (BCE) would remove all clauses from $E$.\\[1ex]
\end{enumerate}
\end{example}~\\[-1ex]
\textbf{Questions:}\\[1ex]
\begin{itemize}\setlength{\itemsep}{1ex}
\item What does BCE do to $F = \bigl\{ \{ o \} \bigr\} \cup E$?
\item What does BCE do to $F = \bigl\{ \{ \lnot o \} \bigr\} \cup E$?
\item What does BCE do to $F = \bigl\{ \{ q \}, \{ \lnot q, o, p \}, \{ \lnot q, \lnot o, \lnot p \} \bigr\} \cup E$?
\end{itemize}
\end{frame}


\begin{frame}{Preprocessing: Relationship with Gate Encodings}
Resolving the clauses of a gate encoding on the output literal $o$ results in a set of tautological clauses.\\[1em]
\begin{block}{Idea: Optimized Variable Elimination for Gate Encodings $E$}
Let a formula $F = E \cup R$ with gate clauses $E$ and remainder $R$ be given.\\
Apply variable elimination as follows:
\vspace*{-1ex}
\begin{align*}
( E_x \cup R_x) \otimes (E_{\overline x} \cup R_{\overline x}) 
%[1ex]
&\equiv (E_x \otimes R_{\overline x}) \cup (R_x \otimes E_{\overline x}) \cup (R_x \otimes R_{\overline x}) \cup (E_x \otimes E_{\overline x})\\[1ex]
%
&\equiv (E_x \otimes R_{\overline x}) \cup (R_x \otimes E_{\overline x}) \cup (R_x \otimes R_{\overline x}) \tag{$E_x \otimes E_{\overline x} \equiv \top$}\\[1ex]
&\equiv (E_x \otimes R_{\overline x}) \cup (R_x \otimes E_{\overline x})
\tag{$(E_x \otimes R_{\overline x}) \cup (R_x \otimes E_{\overline x}) \models R_x \otimes R_{\overline x}$}
\end{align*}
\end{block}~\\[-1ex]
\textbf{Proof Idea:} Each clause $c \in R_x \otimes R_{\overline x}$, derived by resolving $c_x \in R_x$ and $c_{\overline x} \in R_{\overline x}$, can also be derived by resolving clauses in $R_{\overline x} \otimes E_x$ and $E_{\overline x} \otimes R_x$.
\end{frame}

\begin{frame}{Recap.}
\begin{block}{Recap.}
\begin{itemize}
    \item Classic Preprocessing Techniques: Subsumption, Self-subsuming Resolution, Bounded Variable Elimination, Blocked Clause Elimination
    \item Relationship between Preprocessing Techniques and Gate Encodings
\end{itemize}
\end{block}
\begin{block}{Next Time}
Propagation-based Techniques and Proof Checking
\end{block}
\end{frame}

\end{document}
