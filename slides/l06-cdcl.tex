\documentclass[t]{sdqbeamer}
%\documentclass[c]{sdqbeamer} 

\usepackage{listings}
\usepackage{graphicx}
\usepackage{tabularx}
\usepackage{multirow}
\usepackage{multicol}
\usepackage{tabulary}
\usepackage{colortbl}
\usepackage{tikzsymbols}
\usepackage{tikz}
\usetikzlibrary{positioning,fit,shapes}
\usepackage[lined,linesnumbered,ruled,noend]{algorithm2e}
\usepackage{bm}

\hypersetup{
	colorlinks=true,
	urlcolor=kit-orange
}

% set sdqbeamer options
\titleimage{blender-render}
\groupname{Algorithm Engineering}
\grouplogo{ae}
\selectlanguage{english}

% define title etc.pp.
\title[SAT Solving]{Practical SAT Solving}
\subtitle{Lecture 4}
\author{\underline{Markus Iser}, Dominik Schreiber, Tom\'a\v{s} Balyo}
\date{May 06, 2024}

% Existing KIT colors: kit-green, kit-blue, kit-red, kit-gray, kit-orange, kit-lightgreen, kit-brown, kit-purple, kit-cyan
% configure appearance
\setbeamercolor{block title}{bg=kit-blue}
\setbeamercolor{block body}{bg=kit-blue!10}
\setbeamercolor{block title example}{bg=kit-orange}
\setbeamercolor{block body example}{bg=kit-orange!10}
\setbeamertemplate{itemize item}{\color{kit-gray}\textbullet}
\setbeamertemplate{itemize subitem}{\color{kit-gray}\textbullet}
\setbeamercolor{item projected}{bg=kit-gray, fg=kit-gray}
\renewcommand{\insertnavigation}[1]{} % remove navigation bar

% define commands
\definecolor{myblue}{HTML}{0D3B66}
\definecolor{myred}{HTML}{6E0E0A}
\definecolor{mypink}{HTML}{F7B2B7}

\newcommand{\vars}[1]{\textsf{vars} (#1)}
\newcommand{\lits}[1]{\textsf{lits} (#1)}
\newcommand{\clss}[1]{\textsf{clss} (#1)}

\newcommand{\highl}[1]{\textcolor{myblue}{#1}}
\newcommand{\highlo}[1]{\textcolor{myred}{#1}}


\begin{document}
\begin{frame}
	\thispagestyle{empty}
	\titlepage
\end{frame}


\begin{frame}{Results of Feedback Round 1}
\begin{block}{Result 1: Grades}
\begin{itemize}\setlength{\itemsep}{1ex}
    \item Content/Relevance: $2 \times 5p, 4 \times 4p$
    \item Quality: $1 \times 5p, 4 \times 4p, 1 \times 3p$
\end{itemize}
\end{block}
\begin{block}{Result 2: Aspects}
\begin{itemize}\setlength{\itemsep}{1ex}
    \item Positive: practical, discussions, examples, application oriented, exercises, small group size, C++
    \item Negative: C++, ``winner takes all'' in competitions, explanation of sequential counter encoding, technical terms not introduced
\end{itemize}
\end{block}
\begin{block}{Lessons learned}
\begin{itemize}\setlength{\itemsep}{1ex}
    \item Better care of technical terms introduction
    \item Split distribution of bonus points in competitions
    \item Some Python examples in exercises
\end{itemize}
\end{block}
\end{frame}


\begin{frame}{Overview}
    \begin{block}{Recap. Lecture 4: Heuristics and Modern SAT Solving 1}
        \begin{itemize}\setlength{\itemsep}{1em}
            \item Decision Heuristics, Restart Strategies, Phase Saving
            \item Modern SAT Solving 1: Conflict Analysis / Clause Learning
        \end{itemize}
    \end{block}
    \begin{block}{Regarding Lecture 5: Parallel SAT Solving 1}
        This topic will be continued on June 10, 2024: Parallel SAT Solving 2.
    \end{block}
	\begin{block}{Today's Topic: Modern SAT Solving 2}
		\begin{itemize}\setlength{\itemsep}{1ex}
			\item Variable State Independent Decaying Sum (VSIDS) Heuristic
			\item Efficient Unit Propagation, Watched Literals
			\item Clause Forgetting
			\item Preprocessing
		\end{itemize}
	\end{block}
\end{frame}


\begin{frame}{Conflict-driven Clause Learning (CDCL) Algorithm}
\vspace*{-1em}
\begin{columns}[T]
\begin{column}{.3\linewidth}
~\\[1em]
\highl{Last Time}
\begin{itemize}%\setlength{\itemsep}{1em}
    \item Classic Decision Heuristics
    \item Restart Strategies
    \item Clause Learning
    \item Non-Chronological Backtracking
\end{itemize}
\highlo{Today}
\begin{itemize}
    \item Efficient Unit Propagation
    \item Clause Forgetting
    \item Modern Decision Heuristics
    \item Preprocessing
\end{itemize}
\end{column}
\begin{column}{.6\linewidth}
\begin{algorithm}[H]
    \DontPrintSemicolon
    \caption{CDCL(CNF Formula $F$, \&Assignment A $\leftarrow \emptyset$)}
    
    \SetKwFunction{propagation}{\highlo{UNIT PROPAGATION}}
    \SetKwFunction{branching}{\highlo{BRAN}\highl{CHING}}
    \SetKwFunction{conflictanalysis}{\highl{CONFLICT-ANALYSIS}}
    \SetKwFunction{restart}{\highl{RESTART}}
    \SetKwFunction{cleanup}{\highlo{CLEANUP}}
    \SetKwFunction{preprocessing}{\highlo{PREPROCESSING}}

    \SetKwData{UNSAT}{UNSAT}
    \SetKwData{SAT}{SAT}
    
    \lIf {not \preprocessing} {
        \Return \UNSAT
    }
    \While {A is not complete} {
        \propagation\;
        \If {A falsifies a clause in $F$} {
            \lIf {decision level is $0$} {
                \Return \UNSAT
            }
            \Else {
                (clause, level) $\leftarrow$ \conflictanalysis\;
                add clause to $F$ and backtrack to level\;
                \textbf{continue}\;
            }
        }
        \lIf {\restart} {
            backtrack to level $0$
        }
        \lIf {\cleanup} {
            forget some learned clauses
        }
        \branching\;
    }
    \Return \SAT\;
\end{algorithm}
\end{column}
\end{columns}
\end{frame}
    

\definecolor{heat0}{HTML}{f94144}
\definecolor{heat1}{HTML}{f3722c}
\definecolor{heat2}{HTML}{F8961E}
\definecolor{heat3}{HTML}{F9844A}
\definecolor{heat4}{HTML}{F9C74F}
\definecolor{heat5}{HTML}{90BE6D}
\definecolor{heat6}{HTML}{43AA8B}
\definecolor{heat7}{HTML}{4D908E}
\definecolor{heat8}{HTML}{577590}
\definecolor{heat9}{HTML}{277DA1}

\begin{frame}{Unit Propagation: Cost}
\begin{block}{Hot Paths in CDCL Solvers}
\renewcommand{\arraystretch}{1.5}
\begin{tabularx}{\linewidth}{l|l|X}
    \bf heat & $\varnothing~per~sec.$\footnote{Order of magnitude of average event count per second (in runs of Cadical on a large combined benchmark set)} & \\
    \hline
    \cellcolor{heat0} Clause Access & & Unpredictable memory access: most expensive\\
    \cellcolor{heat1} Iterate Occurrences & & Predictable memory access: array of pointers (hardware prefetching)\\ 
    \cellcolor{heat2} \bf Propagation & $\mathbf{\sim 10^6}$ & \bf Access occurrence-list of yet unpropagated literal\\
    \hline
    \cellcolor{heat4} Decision & $\sim 10^3$ & \\
    \cellcolor{heat5} Conflict & $\sim 10^3$ & \it Learn a clause $\rightarrow$ more to check for propagation\\
    \cellcolor{heat6} Restart & $\sim 10^{-1}$ & \\
    \cellcolor{heat7} Cleanup & & \it Forget some learned clauses $\rightarrow$ less to check for propagation
\end{tabularx}
\end{block}
\end{frame}

    
\begin{frame}{Unit Propagation}
\begin{exampleblock}{Example: Unit Propagation with Full Occurrence Lists}
\newcommand{\doublecline}{\cline{2-4}\noalign{\vskip\doublerulesep\vskip-\arrayrulewidth}\cline{2-4}}
\newcommand{\doubleclin}{\cline{2-3}\noalign{\vskip\doublerulesep\vskip-\arrayrulewidth}\cline{2-3}}
\renewcommand{\arraystretch}{1.5}
\begin{columns}
\begin{column}[t]{.3\linewidth}
    \bf Trail\\[1ex]
    \begin{tabularx}{\linewidth}{X|X|X|}
    \multicolumn{1}{l}{\bf level} & \multicolumn{1}{l}{\bf value} & \multicolumn{1}{l}{\bf reason} \\
    \cline{2-3}
    \rowcolor<2>{kit-green} \only<2-4>{1} & \only<2-4>{a} & \only<2-4>{$\bot$} \\
    \doubleclin
    \rowcolor<3-4>{kit-green} \only<3-4>{2} & \only<3-4>{c} & \only<3-4>{$\bot$} \\
    \doubleclin
    \rowcolor<4>{kit-green} \only<4>{2} & \only<4>{b} & \only<4>{\addr{2}} \\
    \cline{2-3}
    \end{tabularx}
\end{column}
\begin{column}[t]{.3\linewidth}
    \bf Occurrence Lists\\[1ex]
    \begin{tabularx}{\linewidth}{X|XXX}
    \multicolumn{1}{l}{\bf idx.} & \multicolumn{3}{l}{\bf occurrences}\\
    \cline{2-4}
    $a$ & \addr{1} & & \\
    \doublecline
    \rowcolor<2>{kit-green} $\lnot a$ & \addr{2} & \addr{3} & \\
    \doublecline
    $b$ & \addr{1} & \addr{2} & \\
    \doublecline
    $\lnot b$ & \addr{3} & & \\
    \doublecline
    $c$	& \addr{3} & \addr{1} & \\
    \doublecline
    \rowcolor<3-4>{kit-green} $\lnot c$ & \addr{2} & ~~~ & \\
    \cline{2-4}
    \end{tabularx}~\\[1em]
\end{column}
\begin{column}[t]{.3\linewidth}
    \bf Formula\\[1ex]
    \begin{tabularx}{\linewidth}{X|XXX|}
    \multicolumn{1}{l}{\bf addr.} & \multicolumn{3}{l}{\bf clause}\\
    \cline{2-4}
    \addr{1} & $a$ & $b$ & $c$ \\
    \doublecline
    \addr{2} & $\lnot a$ & $b$ & $\lnot c$ \\
    \doublecline
    \addr{3} & $\lnot a$ & $\lnot b$ & $c$ \\
    \cline{2-4}
    \end{tabularx}~\\[2em]
\end{column}
\end{columns}
\end{exampleblock}
\end{frame}

\begin{frame}{Unit Propagation: Two Watched Literals}
    \begin{block}{Motivation: Hot Path}
    \renewcommand{\arraystretch}{3}
    \begin{tabularx}{\linewidth}{l|c|L}
        \bf heat & $\varnothing~per~sec.$\footnote{Order of magnitude of average event count per second (in runs of Cadical on a large combined benchmark set)} & 
        \multirow[t]{3}{\linewidth}{%
\textbf{Idea:} Reduced occurrence tracking by only keeping the following \highl{\bf invariant:}\\[1ex]
\highl{\bf Each yet unsatisfied clause is \highlo{watched by}, i.e., in the occurrence list of, \highlo{two of its unassigned literals.}}\\[1ex]
\textbf{Reasoning:} less literals watched $\rightarrow$ shorter occurrence lists $\rightarrow$ less clause accesses $\rightarrow$ fast unit propagation\\[1em]
\begin{itemize}\setlength{\itemsep}{1pt}
    \item<2-> Why do two watched literals per clause suffice?
    \item<3-> Why does one watched literal per clause not suffice?
    \item<4-> How do we keep that invariant? (Branching?, Backtracking?)
\end{itemize}
        }%
    \\
        \cline{1-2}
        \cellcolor{heat0} Clause Access & &\\
        \cellcolor{heat1} Iterate Occurrences & & \\ 
        \cellcolor{heat2} \bf Propagation & $\mathbf{\sim 10^6}$ & \\
        % \hline
        % \cellcolor{heat4} Decision & $\sim 10^3$ & \\
        % \cellcolor{heat5} Conflict & $\sim 10^3$ & \it Learn a clause $\rightarrow$ more to check for propagation\\
        % \cellcolor{heat6} Restart & $\sim 10^{-1}$ & \\
        % \cellcolor{heat7} Cleanup & & \it Forget some learned clauses $\rightarrow$ less to check for propagation
    \end{tabularx}
    \end{block}
\end{frame}

\begin{frame}{Unit Propagation}
    \begin{exampleblock}{Example: Unit Propagation with Two Watched Literals}
    \newcommand{\doublecline}{\cline{2-4}\noalign{\vskip\doublerulesep\vskip-\arrayrulewidth}\cline{2-4}}
    \newcommand{\doubleclin}{\cline{2-3}\noalign{\vskip\doublerulesep\vskip-\arrayrulewidth}\cline{2-3}}
    \renewcommand{\arraystretch}{1.5}
    \begin{columns}
    \begin{column}[t]{.3\linewidth}
        \bf Trail\\[1ex]
        \begin{tabularx}{\linewidth}{X|X|X|}
        \multicolumn{1}{l}{\bf level} & \multicolumn{1}{l}{\bf value} & \multicolumn{1}{l}{\bf reason} \\
        \cline{2-3}
        \rowcolor<2-4>{kit-green} \only<2->{1} & \only<2->{a} & \only<2->{$\bot$} \\
        \doubleclin
        \rowcolor<5,6>{kit-green} \only<5->{2} & \only<5->{c} & \only<5->{$\bot$} \\
        \doubleclin
        \rowcolor<6>{kit-green} \only<6->{2} & \only<6->{b} & \only<6->{\addr{2}} \\
        \cline{2-3}
        \end{tabularx}
    \end{column}
    \begin{column}[t]{.3\linewidth}
        \bf Two Watched Literals\\[1ex]
        \begin{tabularx}{\linewidth}{X|XXX}
        \multicolumn{1}{l}{\bf idx.} & \multicolumn{3}{l}{\bf occurrences}\\
        \cline{2-4}
        $a$ & \addr{1} & & \\
        \doublecline
        \rowcolor<2-4>{kit-green} $\lnot a$ & \only<1-2>{\addr{2}} & \only<1-3>{\addr{3}} & \\
        \doublecline
        $b$ & \addr{1} & \addr{2} & \\
        \doublecline
        $\lnot b$ & \addr{3} & & \\
        \doublecline
        \rowcolor<1,4>{kit-green} $c$ & \only<4->{\addr{3}} & & \\
        \doublecline
        \rowcolor<1,3,5,6>{kit-green} $\lnot c$ & \only<3->{\addr{2}} & & \\
        \cline{2-4}
        \end{tabularx}~\\[1em]
    \end{column}
    \begin{column}[t]{.3\linewidth}
        \bf Formula\\[1ex]
        \begin{tabularx}{\linewidth}{X|XXX|}
        \multicolumn{1}{l}{\bf addr.} & \multicolumn{3}{l}{\bf clause}\\
        \cline{2-4}
        \addr{1} & $a$ & $b$ & $c$ \\
        \doublecline
        \rowcolor<3>{kit-green} \addr{2} & \only<1-2>{$\lnot a$}\only<3>{$\bm{\lnot c$}}\only<4->{$\lnot c$} & $b$ & \only<1-2>{$\lnot c$}\only<3>{$\bm{\lnot a$}}\only<4->{$\lnot a$} \\
        \doublecline
        \rowcolor<4>{kit-green} \addr{3} & \only<1-3>{$\lnot a$}\only<4>{$\bm{c$}}\only<5->{$c$} & $\lnot b$ & \only<1-3>{$c$}\only<4>{$\bm{\lnot a$}}\only<5->{$\lnot a$} \\
        \cline{2-4}
        \end{tabularx}~\\[2em]
    \end{column}
    \end{columns}
    \end{exampleblock}
\end{frame}

\begin{frame}{Unit Propagation: Two Watched Literals}
\vspace*{-1em}
\begin{block}{Two Watched Literals: Optimizations}
\renewcommand{\arraystretch}{3}
\begin{tabularx}{\linewidth}{l|c|L}
    \bf heat & $\varnothing~per~sec.$\footnote{Order of magnitude of average event count per second (in runs of Cadical on a large combined benchmark set)} & 
    \multirow[t]{4}{\linewidth}{%
\textbf{Invariant:} \highl{\bf Each yet unsatisfied clause is watched by two of its unassigned literals.}\\[1ex]
\textbf{$\bm{\rightarrow}$ Reduced Load in Occurrence Tracking}\\[1ex]
\textbf{Optimization 1:} Keep watched literals the first two in clause\\
$\bm{\rightarrow}$ Alternative: Store watched literals in other location\\
{\footnotesize Note: What happens if clauses are kept in shared memory for parallel solving?}\\[1ex]
\textbf{Optimization 2:} Also keep a literal of each clause directly in occurrence list\\
$\bm{\rightarrow}$ Skip clause access if that literal is satisfied 
        }%
    \\
    \cline{1-2}
    \cellcolor{heat0} Clause Access & &\\
    \cellcolor{heat1} Iterate Occurrences & & \\ 
    \cellcolor{heat2} \bf Propagation & $\mathbf{\sim 10^6}$ & 
\end{tabularx}
\end{block}
\end{frame}


\begin{frame}{Recap}
\begin{block}{Unit Propagation}
\begin{itemize}
    \item Hottest path in CDCL solvers
    \item Two watched literals per clause suffice for unit propagation (and conflict detection)
    \item Other optimizations: keep watched literals first in clause, keep a literal of each clause directly in occurrence list
\end{itemize}
\end{block}
\begin{block}{Next Up}
Clause Forgetting
\end{block}
\end{frame}


\begin{frame}{Clause Forgetting}

\begin{block}{Motivation}
Clause learning is most important pruning strategy in CDCL solvers.\\[1ex]
\textbf{Problems:}
\begin{itemize}
\item Without restrictions the number of clauses grows exponentially (Remember: naive Saturation Algorithm)
\item Risk of running out of memory
\item Slows down unit propagation
\end{itemize}
\textbf{Idea:}
\begin{itemize}
\item Periodically forget some learned clauses
\item Keep only important learned clauses
\item Use heuristics to estimate clause importance
\end{itemize}
\end{block}
\end{frame}

\begin{frame}{Clause Forgetting}
\begin{block}{Periodic Clause Forgetting: Heuristics}
\begin{itemize}\setlength{\itemsep}{1em}
    \item \textbf{Clause Size}\\[1ex]
    Keep short clauses
    \item \textbf{Least Recently Used (LRU)}\\[1ex]
    Keep clauses which where reasons in recent conflicts: clause activity (moving average)
    \item \textbf{Literal Block Distance (LBD)}\\[1ex]
    Keep clauses with a low number of decision levels%
    \footnote{\href{https://www.ijcai.org/Proceedings/09/Papers/074.pdf}{Predicting Learnt Clauses Quality in Modern SAT Solvers, Audemard \& Simon (IJCAI 2009)}}
\end{itemize}
\end{block}
\end{frame}
    
    
\begin{frame}{Forgetting Heuristic: Literal Block Distance (LBD)}
\href{http://www.cril.univ-artois.fr/articles/communities.pdf}{``Impact of Community Structure on SAT Solver Performance'', Newsham et al., SAT 2014}
\begin{block}{Take home: LBD correlates with number of touched communities}
\centering
\centering\includegraphics[width=.9\linewidth]{figures/l06/community-structure-and-learning.png}
\vspace*{-1em}
\begin{flushright}
    \footnotesize
    Image Source: \href{https://arxiv.org/pdf/1606.03329}{``Community Structure in Industrial SAT Instances'', Ansotegui et al., AIJ 2019}
\end{flushright}
\end{block}
\end{frame}
    
    
\begin{frame}{Clause Forgetting: Modern Hybrid Approach}
\begin{block}{Three-Tier Clause Management}
    Manage clauses differently in three tiers: 
    \begin{itemize}
    \item \texttt{core, LBD}: permanently store clauses of LBD $\leq k$ (core-cut value, $3$ in practice)
    \item \texttt{mid-tier, LRU}: clauses stay here if used in recent conflicts
    \item \texttt{local, LRU}: keep fixed number of clauses (say $5000$) of highest activity
    \end{itemize}
\end{block}
\begin{block}{History}
    \begin{itemize}
    \item \texttt{core} and \texttt{local} tier introduced in SWDiA5BY (Chanseok Oh, 2014)
    \item \texttt{mid-tier} introduced in CoMinisatPS (Chanseok Oh, 2015)
    \item \href{https://link.springer.com/chapter/10.1007/978-3-319-24318-4_23}{``Between SAT and UNSAT: The Fundamental Difference in CDCL SAT'' (Chanseok Oh, 2015)}
    \item Note: MapleCOMSPS (2016) is a CoMinisatPS fork
    \end{itemize}
\end{block}
\end{frame}

% \begin{frame}{More on Clause Learning}
% \begin{block}{``Clause Size Reduction with all-UIP Learning''}
%     \begin{itemize}
%     \item Feng \& Bacchus, SAT 2020
%     \item 1-UIP clause has smallest LBD: continue conflict resolution as long as LBD does not increase but clause size decreases
%     \item SAT Competition 2020: \texttt{CaDiCaL AllUIP} (won planning track)
%     \end{itemize}
% \end{block}
% \end{frame}

\begin{frame}{Recap}
\begin{block}{So far}
\begin{itemize}
    \item Efficient Unit Propagation
    \item Clause Forgetting Heuristics:
    \begin{itemize}
        \item Size, LRU, LBD
        \item LBD correlation with communities
        \item Three-Tier Clause Management
    \end{itemize}
\end{itemize}
\end{block}
\begin{block}{Next Up}
Modern Branching Heuristics
\end{block}
\end{frame}

    
\begin{frame}{{\bf V}ariable {\bf S}tate-{\bf I}ndependent {\bf D}ecaying {\bf S}um {\bf (VSIDS)}}
    Implemented in most CDCL solvers.
    First presented in SAT solver Chaff.\footnote{Chaff: Engineering an efficient SAT solver (Moskewicz et al., 2001)}
    \begin{block}{VSIDS Heuristic}
    Compute score for each variable, select variable with highest score:\\[1em]
    \begin{itemize}\setlength{\itemsep}{1em}
        \item Initialize variable score (with zero or use some static heuristic)
        \item New conflict clause $c$: Score is incremented for all variables in $c$
        \item Periodically, divide all scores by a constant
    \end{itemize}	
    \end{block}
    \end{frame}
    
    \begin{frame}{VSIDS Heuristic}
    \begin{exampleblock}{VSIDS Example}
    \begin{small}
    \begin{minipage}{5cm}
    \textbf{Initial $F$:}\\
    $\{ x_1, x_4 \}$ \\
    $\{ x_1, \overline{x_3}, \overline{x_8} \}$ \\
    $\{ x_1, x_8, x_{12} \}$ \\
    $\{ x_2, x_{11} \}$ \\
    $\{ \overline{x_7}, \overline{x_3}, x_9 \}$ \\
    $\{ \overline{x_7}, x_8, \overline{x_9} \}$ \\
    $\{ x_7, x_8, \overline{x_{10}} \}$ \\
    \\
    \\
    \textbf{Scores:}\\
    $4: x_8$ \\
    $3: x_1, x_7$ \\
    $2: x_3$ \\
    $1: x_2, x_4, x_9, x_{10}, x_{11}, x_{12}$
    \end{minipage}
    %
    \only<2->{
    \begin{minipage}{6cm}
    \textbf{$F$ with new learned clause added:}\\
    $\{ x_1, x_4 \}$ \\
    $\{ x_1, \overline{x_3}, \overline{x_8} \}$ \\
    $\{ x_1, x_8, x_{12} \}$ \\
    $\{ x_2, x_{11} \}$ \\
    $\{ \overline{x_7}, \overline{x_3}, x_9 \}$ \\
    $\{ \overline{x_7}, x_8, \overline{x_9} \}$ \\
    $\{ x_7, x_8, \overline{x_{10}} \}$ \\
    {\color{mypink} $\{ x_7, x_{10}, \overline{x_{12}} \}  \quad $} (new learned clause)\\
    \\
    \textbf{Scores:}\\
    $4: x_8, {\color{mypink} x_7}$ \\
    $3: x_1$ \\
    $2: x_3, {\color{mypink} x_{10}}, {\color{mypink} x_{12}} $ \\
    $1: x_2, x_4, x_9,  x_{11}$
    \end{minipage}
    }
    \end{small}
\end{exampleblock}
\end{frame}
    
    
    \begin{frame}{VSIDS Heuristic}
    
    Keep list of variables sorted by score
    
    \begin{block}{VSIDS Common Implementation: Binary Heap}
    \begin{itemize}
    \item Backtrack: \texttt{insert\_with\_priority} $\in \mathcal{O}(\log n)$
    \item Branch: \texttt{pull\_highest\_priority\_element} $\in \mathcal{O}(\log n)$
    \item Bump: \texttt{increase\_key} $\in \mathcal{O}(\log n)$
    \item Decay: \texttt{decay} $\in \mathcal{O}(n)$
    \end{itemize}
    \end{block}
    
    \begin{block}{Reasoning behind VSIDS} Make heuristics more ``focused''
        \begin{itemize}
            \item try to find small unsatisfiable subsets
            \item prefer variables that occurred in a recent conflict
        \end{itemize}
    \end{block}
    \end{frame}
    
    
    \begin{frame}{VSIDS Heuristic}
    Periodically divide scores to give priority to recently learned clauses
    \begin{exampleblock}{VSIDS Variants}
    \begin{itemize}
      \item Chaff (2001)
      \begin{itemize}
      \item half scores every 256 conflicts (``decay'')
      \item sort priority queue after each decay only
      \end{itemize}
      
      \item Berkmin (2002) 
      \begin{itemize}
      \item bump all literals in implication graph
      \item divide scores by 4
      \end{itemize}
      
      \item Minisat (2003)
      \begin{itemize}
      \item {\color{myblue} exponential VSIDS} (eVSIDS)
      \end{itemize}
    \end{itemize}
    \end{exampleblock}
    
    \begin{block}{Alternatives}
      \begin{itemize}
          \item Siege (2004): Variable Move To Front (VMTF)
        \item HaifaSAT (2008): Clause Move To Front (CMTF)
      \end{itemize}
    \end{block}
    \end{frame}
    
\begin{frame}{Comparison of Heuristics}%
\begin{block}{Evaluating CDCL Variable Scoring Schemes (Biere \& Fröhlich, 2015)}
    \centering
    \includegraphics[width=.6\linewidth]{figures/l06/evsids-vmtf-acids-froehlich2015.png}
\end{block}
\end{frame}
    
\begin{frame}{Recent Hybrid Approaches}
\begin{block}{Hybrid Approaches} 
\begin{itemize}\setlength{\itemsep}{1em}
\item {\bf Warmup Phase}:
\begin{itemize}\setlength{\itemsep}{1ex}
    \item MapleCOMSPS (2016): use Learning Rate-based Branching (LRB) in \emph{initial} period, then switch to VSIDS
    \item Maple\_LCM\_Dist (2017): use Distance Heuristic (Dist.) in \emph{initial} period, then switch to VSIDS
\end{itemize}
\item {\bf Reinforcement Learning}: Kissat\_MAB (2021)
\begin{itemize}\setlength{\itemsep}{1ex}
    \item Two-armed Bandid switches between VSIDS and Conflict History-Based (CHB) Heuristic
    \item Reward function favors variables that contribute to learning ``good'' clauses
    \end{itemize}
\end{itemize}
\end{block}
\end{frame}


\begin{frame}{Recap}
    \begin{block}{So far}
        \begin{itemize}
            \item Unit Propagation
            \item Clause Forgetting
            \item Modern Branching Heuristics
            \begin{itemize}
                \item Mostly VSIDS
                \item Hybrid approaches: warmup VSIDS scores, reinforcement learning
            \end{itemize}
        \end{itemize}
    \end{block}
    \begin{block}{Next Up}
        Evolution of formula structure with clause learning
    \end{block}
\end{frame}


% \begin{frame}{Hybrid Backtracking}
%     \begin{block}{Recent Revival of Chronological Backtracking}
%     \begin{itemize}
%     \item Nadel \& Ryvchin, SAT 2018
%     \item Backtrack chronologically iff number of untouched decision levels is higher than a given bound
%     \item SAT Competition 2018: \texttt{MapleLCMDistChronoBT} (best score)
%     \item SAT Race 2019: \texttt{MapleLCMDistChronoBT} \& \texttt{CaDiCal} (best scores)
%     \item Hard to implement: Decision levels no longer monotonically increasing on trail
%     \end{itemize}
%     \end{block}
% \end{frame}

\graphicspath{ {satvizpdf/} }

\begin{frame}{Instance: Aprove (Termination Analysis, SAT)}{
    \only<1>{initial layout, recently active variables after 1000 conflicts}
    \only<2>{initial layout, recently active variables after 1690 conflicts}
    \only<3>{initial layout, recently active variables after 3090 conflicts}
    \only<4>{initial layout, recently active variables after 5000 conflicts}
    \only<5>{relayout after 6000 conflicts}
    \only<6>{core after 52500 conflicts}
}
\centering
\vspace*{-1em}
\only<1>{
\includegraphics[width=0.9\textwidth, trim={100 0 100 100}, clip]{satviz-aprove-1000}
}
\only<2>{
\includegraphics[width=0.9\textwidth, trim={100 0 100 100}, clip]{satviz-aprove-1690}
}
\only<3>{
\includegraphics[width=0.9\textwidth, trim={100 0 100 100}, clip]{satviz-aprove-3090}
}
\only<4>{
\includegraphics[width=0.9\textwidth, trim={100 0 100 100}, clip]{satviz-aprove-5000}
}
\only<5>{
\includegraphics[width=0.9\textwidth, trim={100 0 100 100}, clip]{satviz-aprove-6000-relayout}
}
\only<6>{
\includegraphics[width=0.9\textwidth, trim={100 0 100 100}, clip]{satviz-aprove-52500-zoom}
}
\end{frame}

\begin{frame}{Instance: Newton SMT (SV Competition, SAT)}{
    \only<1>{initial layout}
    \only<2>{after 10000 conflicts}
    \only<3>{after 1000000 conflicts}
    \only<4>{after 3000000 conflicts}
    \only<5>{core, after 3500000 conflicts, almost solved}
}
\centering
\vspace*{-2em}
\only<1>{
\includegraphics[width=0.9\textwidth, trim={100 0 100 100}, clip]{newton-smt-5200}
}
\only<2>{
\includegraphics[width=0.9\textwidth, trim={100 0 100 100}, clip]{newton-smt-10000}
}
\only<3>{
\includegraphics[width=0.9\textwidth, trim={100 0 100 100}, clip]{newton-smt-1000000}
}
\only<4>{
\includegraphics[width=0.9\textwidth, trim={100 0 100 30}, clip]{newton-smt-3000000}
}
\only<5>{
\includegraphics[width=0.9\textwidth, trim={100 0 100 100}, clip]{newton-smt-3500000-cherry}
}
\end{frame}

    
\begin{frame}{Preprocessing}
    \begin{block}{General Idea}
      Conjecture: Smaller problems are easier to solve\\[1em]
      $\Longrightarrow$ Try to reduce the size of the input formula by (polynomial time) simplification procedures.
    \begin{itemize}
      \item Subsumption
      \item Self-subsuming resolution
      \item Variable elimination
      \item Blocked clause elimination
      \item Failed literal elimination
      \item Autarkies
    \end{itemize}
    \end{block}
    \end{frame}
    
    
    \begin{frame}{Subsumption}
    \begin{block}{Subsumption}
    \begin{itemize}
    \item Definition: A clause $D$ subsumes a clause $C$ iff $D \subseteq C$
    \item We also say that clause $C$ is subsumed by $D$
    \item To check satisfiability, subsumed clauses are irrelevant: $\forall D \subseteq C, D \models C$
    \end{itemize}
    \end{block}
    
    \begin{exampleblock}{Example}
    $\{a, b\}$ subsumes $\{a, b, c\}$ and $\{a, b, d\}$
    \end{exampleblock}
    \end{frame}
    
    
\begin{frame}{Self-Subsuming Resolution}
Combination of Subsumption and Resolution
\begin{block}{Self-Subsuming Resolution}
    \begin{itemize}
    \item Definition: Let $C, D$ be clauses and $\otimes_x$ the resolution operator on variable $x$. If $C \otimes D \subseteq C$ then $C$ is said
    to be \emph{self-subsumed by $D$ with respect to $x$}.
    \item The resolvent of $C$ and another clause $D$ subsumes $C$
    \end{itemize}
\end{block}
   
\begin{exampleblock}{Example}
    \begin{itemize}
    \item $C := \{ \lnot b, \lnot e, {\color{myblue} f}, \lnot h \}$
    \item $D := \{ \lnot b, \lnot e, {\color{myblue} \lnot f} \}$
    \item $E := C \otimes_f D = \{ \lnot b, \lnot e, \lnot h \}$
    \item The clause $C$ is subsumed by the resolvent $E$
    \end{itemize}
\end{exampleblock}
\end{frame}
    
    
    \begin{frame}{Variable Elimination}
    \begin{block}{Variable Elimination}
    \begin{itemize}
    \item Definition:
    \begin{itemize}
    \item Let $S_x, S_{\overline x} \subset F$ be the sets of clauses containing $x$ resp. ${\overline x}$
    \item Let $R = \{ C \otimes_x D ~|~ C \in S_x, D \in S_{\overline x} \}$ be the set of all resolvents on $x$
    \item Modify $F$ such that $F' := (F \setminus (S_x \cup S_{\overline x})) \cup R$
    \end{itemize}
    \item The formulas $F$ and $F'$ are \emph{satisfiability equivalent} (not equivalent)
    \end{itemize}
    \end{block}
    
    \begin{block}{Bounded Variable Elimination (BVE)}
    \begin{itemize}
    \item Simulate variable elimination, but replace clauses only if number of clauses decreases (due to tautological resolvents) 
    \item Most important simplification technique today
    \end{itemize}
    \end{block}
    \end{frame}
    
    
    \begin{frame}{Blocked Clauses}
    \begin{block}{Definition}
    A clause $(l \lor C)$ is blocked in $F$ by $l$ if either $l$ is \emph{pure} in $F$ or if for every clause $(\lnot l \lor D)$ in $F$ the resolvent $(C \lor D)$ is a \emph{tautology}.
    \end{block}
    
    \begin{exampleblock}{Example}
    $F := (a \lor b) \land (a \lor \lnot b \lor \lnot c) \land (\lnot a \lor c)$
    
    First clause is not blocked, second is blocked by both $a$ and $\lnot c$, third is blocked by $c$.
    \end{exampleblock}
    
    \begin{block}{Blocked Clause Elimination}
    Removal of an arbitrary blocked clause preserves satisfiability. Blocked clause elimination (BCE) has a unique fixpoint.
    \end{block}
    \end{frame}
    
\begin{frame}{Blocked Clauses and Gates}
\begin{block}{Left-Totality of Gate Encodings}
    Given a gate~$G$ with output variable $o$ and its clausal encoding $E$, 
    it holds that for every clause $C \in E$ either $o \in C$ or $\overline o \in C$ (part~1)
    and all resolvents in $E[o] \otimes_o E[\overline o]$ are tautologic (part~2).
    \end{block}
    
    \begin{block}{Proof of Part 1}
    Assume that there is a clause $C \in E$ such that $o \not\in \vars{C}$. 
    It follows that there exists an assignment to input variables which falsifies $C$ for any assignment to $o$. 
    This contradicts left-totality.
    \end{block}
    
    \begin{block}{Proof of Part 2}
    Let $R$ be a non-tautological resolvent in $E[o] \otimes_o E[\overline o]$. 
    By Definition of resolution, it holds that $o \not\in \vars{R}$ and $E \models R$. 
    This contradicts left-totality. 
\end{block}
\end{frame}
    
    
\begin{frame}{Variable Elimination and Gates}
\begin{block}{Property of Gate Encodings}
    \begin{itemize}
    \item Resolving the clauses of a gate results in tautological clauses
    \item Example: Tseitin encoding of gate $x = \mathop{AND}(y,z)$ results in the clauses $\{ \lnot x, y \}, \{ \lnot x, z \}, \{ x, \lnot y, \lnot z \}$
    \end{itemize}
\end{block}

\begin{block}{Idea: Variable Elimination for Gate Encoding $G$}
    \begin{itemize}
    \item Split formula $F$ into $F = G \cup R$, where $G$ are the gate clauses
    \item Apply variable elimination:\vspace*{-.7em}
    \begin{align*}
    S' &= (G_x \cup R_x) \otimes (G_{\overline x} \cup R_{\overline x})\\
       &= (G_x \otimes R_{\overline x}) \cup (R_x \otimes G_{\overline x}) \cup (R_x \otimes R_{\overline x}) \cup \hcancel{(G_x \otimes G_{\overline x})}\\
       &= (G_x \otimes R_{\overline x}) \cup (R_x \otimes G_{\overline x}) \cup \hcancel{(R_x \otimes R_{\overline x})}\\
       &= (R_x \otimes G_{\overline x}) \cup (R_{\overline x} \otimes G_x)
    \end{align*}\vspace*{-2em}%
    \item $(G_x \otimes R_{\overline x}) \cup (R_x \otimes G_{\overline x}) \models (R_x \otimes R_{\overline x})$ {\color{mypink} (Why? $\rightarrow Exercise$)}
    \end{itemize}
\end{block}
\end{frame}
    

\begin{frame}{Failed Literal Elimination}
\begin{block}{Definition}
If $F \land \{ l \} \vdash_{\mathop{UP}} \bot$, i.e., applying unit propagation on $F \land \{ l \}$ derives UNSAT, replace $F$ by $F \land \{ \lnot l\}$.
\end{block}
    
\begin{block}{Generalization} 
If $(F \setminus \{ C \}) \land \lnot C \vdash_{\mathop{UP}} \bot$, remove $C$ from $F$.
\end{block}
\end{frame}

    
\begin{frame}{Autarkies}
\begin{block}{Definition}
Given a partial assignment $\sigma$ and a formula $F$, a clause $C \in F$ is \emph{touched by $\sigma$} if it contains the negation of a literal 
assigned in $\sigma$. A clause  is \emph{satisfied by $\sigma$} if it contains a literal assigned to $\true$ by $\sigma$.
If all touched clauses are satisfied then $\sigma$ is an \emph{autarky}.
\end{block}

All clauses touched by an autarky can be removed.

\begin{exampleblock}{Autarky-based Clause Removal}
The partial assignment $\sigma = \{ \lnot a, \lnot c \}$ is an autarky for $F := \{ \lnot a, b \}, \{ \lnot a, c \}, \{ a, \lnot b, \lnot c \}, \{ b, d \}, \dots$ (more clauses without $a$ and $c$)
\end{exampleblock}
\end{frame}
    
    
\begin{frame}{Preprocessing Techniques that do not Reduce the Problem Size}
\begin{itemize}
\item Bounded Variable Addition (BVA) a.k.a. Extension rule
\item Some formulas have refutations of exponential size in the resolution calculus, but of polynomial size in extended resolution, e.g., pigeonhole formulas, mutilated chessboard, \dots
\end{itemize}
\begin{block}{Extended Resolution}
Extended resolution adds a second rule to the resolution calculus, the Extension Rule. The idea is to introduce
new variables as conjunction of existing literals, $x_\mathrm{new} \leftrightarrow l_1 \land l_2$. As a rule for formulas in CNF:
\vspace{-2ex}
$$ \cfrac{}{(\lnot x_\mathrm{new} \lor l_1) \land (\lnot x_\mathrm{new} \lor l_2) \land (x_\mathrm{new} \lor \lnot l_1 \lor \lnot l_ 2) } $$
\end{block}
\end{frame}
    

\begin{frame}{Inprocessing}

\begin{block}{Idea: Interleave search and preprocessing}
\begin{itemize}
\item Preprocessing can be extremely beneficial
\item Most solvers in SAT competitions use bounded variable elimination, subsumption and self-subsuming resolution
\item Problem: Many preprocessing techniques, though polynomial, require considerable time
\item Possible Solutions:
\begin{itemize}
\item Interrupt preprocessing techniques after some time
\item Resume preprocessing between restarts
\item Limit preprocessing time in relation to search
\end{itemize}
\end{itemize}
\end{block}
\end{frame}




\begin{frame}{The End.}
\begin{block}{Recap}
    \begin{itemize}\setlength{\itemsep}{1em}
        \item 
    \end{itemize}
\end{block}
\end{frame}

\end{document}
